	
\newcommand{\CLASSINPUTbaselinestretch}{1.0} % baselinestretch
\newcommand{\CLASSINPUTinnersidemargin}{0.75in} % inner side margin
\newcommand{\CLASSINPUToutersidemargin}{0.75in} % outer side margin
\newcommand{\CLASSINPUTtoptextmargin}{0.75in}   % top text margin
\newcommand{\CLASSINPUTbottomtextmargin}{0.75in}% bottom text margin

\newcommand{\revised}[1]{{\color{blue}#1}}

\documentclass[conference]{IEEEtran}

\IEEEoverridecommandlockouts % precisa para usar o \thanks{}
% *** CITATION PACKAGES ***
%
\usepackage{cite}
\usepackage{flushend}
\usepackage{xcolor}
\usepackage[pdftex]{graphicx}

\usepackage[utf8]{inputenc}


% *** MATH PACKAGES ***
%
%\usepackage[cmex10]{amsmath}
\usepackage{amsmath}
\usepackage{amssymb}

% *** SPECIALIZED LIST PACKAGES ***
%
\usepackage{algorithm}
\usepackage{algorithmic}

% *** ALIGNMENT PACKAGES ***
%
\usepackage{array}
\usepackage{multirow}

% *** SUBFIGURE PACKAGES ***
\usepackage[tight,footnotesize]{subfigure}

% *** PDF, URL AND HYPERLINK PACKAGES ***
%
%\usepackage[bookmarks=false]{hyperref}
\usepackage{url}

% correct bad hyphenation here
\hyphenation{rep-re-sen-ta-tion fe-de-ral}


\begin{document}
%
% paper title
% can use linebreaks \\ within to get better formatting as desired
\title{\vspace{0.25in}Avaliação estatística sobre seleção de características para categorização de texto}


% author names and affiliations
% use a multiple column layout for up to three different
% affiliations

\author{\IEEEauthorblockN{
		Rog\'{e}rio C. P. Fragoso\IEEEauthorrefmark{1}, 
		Lucas F. Melo\IEEEauthorrefmark{1},
		Saulo C. R. P. Sobrinho\IEEEauthorrefmark{1}
}
\IEEEauthorblockA{Universidade Federal de Pernambuco (UFPE), Centro de Inform\'{a}tica (CIn)\\
Av. Jornalista Anibal Fernandes s/n, Cidade Universit\'{a}ria 50740-560, Recife, PE, Brazil\\
rcpf@cin.ufpe.br, lfm2@cin.ufpe.br, scrps@cin.ufpe.br \\}}
%\IEEEauthorrefmark{1}Telephone: +55 81 2126-8430 Ext.4346 Fax: +55 81 2126-8438}}

% make the title area
\maketitle

\begin{abstract}
	%\boldmath
\textcolor{red}{ESCREVER RESUMO}
\end{abstract}

\IEEEpeerreviewmaketitle

% Formato descrito na página da disciplina. Lá ela diz que a presença destas seções é mandatória. 
% (Acho que a gente não precisa organizar nesse formato exatamente, mas precisamos garantir que cada item esteja presente)

%•       Passo 1 – Justificativa: De início, explicitam-se os motivos que justificam a pesquisa, determinando-se e delimitando-se o problema, o qual deve estar formulado de maneira clara e precisa.
%•       Passo 2 - Fundamentação teórica: Descreve-se  o relacionamento do problema com a teoria que será utilizada na pesquisa.
%•       Passo 3 - Objetivo da pesquisa: Os objetivos devem ser retirados diretamente dos problemas levantados no Passo 2. Define-se o que se pretende alcançar com a realização do trabalho.
%•       Passo 4 - Especificação da amostra: Deve-se determinar a área de execução da pesquisa, a população a ser investigada, o tipo de amostra e a determinação do seu tamanho, bem como o tipo de amostragem a ser utilizado. Define-se as variáveis envolvidas.
%•       Passo 5– Análise exploratória: Fazer um estudo descritivo dos dados (gráficos e medidas).  Verificar normalidade dos dados e potenciais pontos aberrantes.
%•       Passo 6– Metodologia (Formulação das hipóteses): Estabelecem-se as hipóteses que serão formuladas, as quais devem ser claras e precisas. Define-se o problema estatisticamente, decidindo-se que informação estatística é realmente necessária e qual método que será aplicado.   
%•       Passo 7 - Análise dos resultados: Passa-se ao tratamento dos dados por intermédio dos testes estatísticos, os quais dependem das hipóteses que serão testadas. Nesse tópico, é exigido que sejam aplicados testes de hipóteses paramétricos e/ou não paramétricos. Testes de duas amostras são exigidos, quando comparando abordagens.

\section{Introdução}
\label{sec:intro}

\textcolor{red}{Nesta seção, apresentar o problema de seleção de características, categorização de textos e os conceitos básicos necessários para o entendimento do trabalho.}

\subsection{Fundamentação teórica}
%Seguindo o modelo sugerido no site, essa seção é obrigatória.

Esta seção apresentou conceitos básicos de categorização de textos e seleção de características. 
O restante do trabalho é organizado como segue: 
A Seção \ref{sec:objetivo} apresenta o objetivo do presente trabalho. 
Na Seção \ref{sec:exp} são detalhadas as configurações dos experimentos, incluindo descrição da base de dados, os algoritmos de interesse e as hipóteses a serem verificadas sobre os dados. 
A Seção \ref{sec:analise} demonstra os procedimentos estatísticos realizados no trabalho.
Finalmente, a Seção \ref{sec:conclusao} apresenta as conclusões do trabalho.

\section{Objetivo}
\label{sec:objetivo}

O objetivo do presente trabalho é avaliar o desempenho de 5 métodos de seleção de características usados em categorização de texto. Para tanto, executamos análises estatísticas para avaliar a aderência dos dados amostrais a uma distribuição normal. Posteriormente, aplicamos testes de hipóteses adequados para, efetivamente, comparar os desempenho dos métodos.


\section{Experimentos}
\label{sec:exp}

Esta seção descreve as configurações dos experimentos.

\subsection{Base de dados}
\label{sec:bd}

Neste trabalho a base de dados \textit{Reuters 10} foi utilizada. Esta base de dados é um subconjunto da coleção \textit{Reuters-21578}~\footnote{Disponível em http://disi.unitn.it/moschitti/corpora.htm.}, que é uma das bases
mais utilizadas em trabalhos de categorização de texto. A base é
composta por documentos coletados do \textit{Reuters newswire} de 1987 e apresenta 135
categorias. Entretanto, neste trabalho foi adotado um subconjunto composto pelas
10 maiores categorias da base. O subconjunto \textit{Reuters 10} contém 9.980 documentos
e seu vocabulário abarca 10.987 termos. A base de dados \textit{Reuters 10} também
é bastante utilizada em trabalhos de categorização de texto~\cite{chang2008multilabel,chen2009feature,yang2011new}. 

A distribuição dos documentos é bastante desbalanceada,
apresentando categorias representando desde 2,3\% até 39\% do tamanho total da
base. Nesta base foram aplicados os seguintes procedimentos de pré-processamento:
remoção termos com duas ou menos letras, remoção de \textit{stopwords} e \textit{stemming}, com o algoritmo \textit{Iterated Lovins
Stemmer}~\cite{lovins1968development}.


\subsection{Metodologia}
\label{sec:metodologia}

A validação cruzada estratificada foi utilizada como método para estimativa de acurácia. Esta técnica é adotada para avaliar a capacidade de generalização de um modelo, a partir de um conjunto de dados. Neste trabalho utilizou-se a variação validação cruzada estratificada com \emph{10 folds}, na qual a base de dados $\mathcal{D}$ é particionada em 10 subconjuntos (\emph{folds}), de tamanhos semelhantes, mantendo a proporção de documentos por categorias equivalente à proporção encontrada no conjunto original. Então, são construídos 10 classificadores, cada um utilizando uma parcela dos \emph{folds} para treinamento e outra parcela para realizar o teste do mesmo, de modo a gerar diferentes combinações dos \emph{folds}. A avaliação final é dada pela média das medidas obtidas em cada uma das 10 execuções~\cite{kohavi1995study}. Nos experimentos realizados com os métodos MFD, MFDR e cMFDR, nove partições foram utilizadas para treinamento e uma partição foi utilizada para teste. Os experimentos executados com AFSA utilizaram oito partições para treinamento, uma para validação e uma para teste.

Assim, temos dez medidas de desempenho para cada um dos quatro métodos de seleção de características avaliados. Esses dados correspondem às entradas para as análises estatísticas realizadas neste trabalho.

A medida de desempenho utilizada nos experimentos foi \textit{Micro-F1}. Seu cálculo é dado pela Eq.~\ref{eq:micro_f1}.

\begin{equation}
\operatorname{\mathcal{F}{1} = \frac{2 x \mathcal{P} x \mathcal{R}}{\mathcal{P} + \mathcal{R}}},
\label{eq:micro_f1}
\end{equation}

\noindent onde $\mathcal{P}$ é uma medida chamada precisão e $\mathcal{R}$ é cobertura~\cite{chang2008multilabel}. As fórmulas para calcular a precisão $\mathcal{P}$ e a cobertura $\mathcal{R}$ são exibidas a seguir.

\begin{equation}
\operatorname{\mathcal{P}} = \frac{\sum_{j=1}^{C}TP_j}{\sum_{j=1}^{C}(TP_j + FP_j)}
\label{eq:precision}
\end{equation}

\begin{equation}
\operatorname{\mathcal{R}} = \frac{\sum_{j=1}^{C}TP_j}{\sum_{j=1}^{C}(TP_j + FN_j)}
\label{eq:recall}
\end{equation}

$TP_j$ é a quantidade de instâncias corretamente rotuladas como pertencentes à categoria $c_j$, $FP_j$ é a quantidade de instâncias incorretamente rotuladas como pertencentes à categoria $c_j$ e $FN_j$ é a quantidade de instâncias incorretamente rotuladas como não pertencentes à categoria $c_j$. 

\section{Análise estatística}
\label{sec:analise}

\subsection{Estatística descritiva}

Uma boa prática ao iniciar uma análise de um conjunto de dados, e que é sugerida por muitos autores, é o uso de técnicas de estatística descritiva para visualização dos dados \cite{montgomery2010applied}.

\begin{table}[h]
	\centering
	\caption{Estatística descritiva}
	\label{tab:est_descr}
	\begin{tabular}{cccc}
		Método    & Média  & Mediana & Desv. Padrão  \\
		\hline
		AFSA&		81.05262 	& 80.9594 	& 0.9634384 \\
		cMFDR&      81.39266 	& 81.38469 	& 1.027409 \\
		MFDR& 		79.03808 	& 79.05812 	& 1.119577 \\
		MFD&     	81.93872 	& 81.91752 	& 1.029598 \\
		\hline
	\end{tabular}
\end{table}

\textcolor{red}{FALAR SOBRE OS RESULTADOS
Quais aparentam ser normais (média aprox. igual à mediana)
}


A Figura~\ref{fig:boxplot} apresenta uma.....

\begin{figure}[h]
	\centering
	\includegraphics[width=\linewidth]{img/boxplot.pdf}
	\caption{\textcolor{red}{Escrever uma descrição}.}
	\label{fig:boxplot}
\end{figure}

As Figuras~\ref{fig:hist_afsa} a~\ref{fig:hist_mfd} apresentam os histogramas das amostras.....

\begin{figure}[h]
	\centering
	\includegraphics[width=\linewidth]{img/hist_afsa.pdf}
	\caption{\textcolor{red}{Escrever uma descrição}.}
	\label{fig:hist_afsa}
\end{figure}

\begin{figure}[h]
	\centering
	\includegraphics[width=\linewidth]{img/hist_cmfdr.pdf}
	\caption{\textcolor{red}{Escrever uma descrição}.}
	\label{fig:hist_cmfdr}
\end{figure}

\begin{figure}[h]
	\centering
	\includegraphics[width=\linewidth]{img/hist_mfdr.pdf}
	\caption{\textcolor{red}{Escrever uma descrição}.}
	\label{fig:hist_mfdr}
\end{figure}

\begin{figure}[h]
	\centering
	\includegraphics[width=\linewidth]{img/hist_mfd.pdf}
	\caption{\textcolor{red}{Escrever uma descrição}.}
	\label{fig:hist_mfd}
\end{figure}




\textcolor{red}{FAZER UMA CONCLUSÃO DA SEÇÃO
Falar das impressões sobre a normalidade dos dados com bases nos histogramas, boxplot e medidas estatísticas.......}

\subsection{Testes de aderência}

\textcolor{red}{Escrever uma motivação para o uso dos testes de aderência e uma breve explicação de como funcionam}

A Tabela~\ref{tab:aderencia} exibe os resultados dos testes Shapiro-Wilk e Kolmogrov-Smirnov para os quatro métodos.

\begin{table}[]
	\centering
	\caption{Resultados dos testes de aderência}
	\label{tab:aderencia}
	\begin{tabular}{c|ccc}
		\hline
		& \multicolumn{2}{c}{p-value}      \\
		\cline{2-3}
		& Shapiro-Wilk & Kolmogrov-Smirnov \\
		\hline
		AFSA  & 0.2353       & 0.1818            \\
		cMFDR & 0.3107       & 0.1818            \\
		MFDR  & 0.6773       & 0.1818            \\
		MFD   & 0.3597       &  0.1818               \\
		\hline
	\end{tabular}
\end{table}

\textcolor{red}{Escrever sobre os resultados do test Shapiro-Wilk}

\textcolor{red}{Escrever sobre os resultados do test Kolmogrov-Smirnov}

\subsection{Testes de hipóteses}

\textcolor{red}{Ainda não executei os testes de hipóteses}

\section{Conclusão}
\label{sec:conclusao}

\textcolor{red}{Escrever conclusão}



\bibliographystyle{IEEEtran}
% argument is your BibTeX string definitions and bibliography database(s)
% \bibliography{IEEEabrv,../bib/paper}
\bibliography{bb}

\end{document}